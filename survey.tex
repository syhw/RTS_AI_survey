\documentclass[journal]{IEEEtran}

\usepackage{cite,color,graphicx}

% *** GRAPHICS RELATED PACKAGES ***
%
\ifCLASSINFOpdf
  % \usepackage[pdftex]{graphicx}
  % declare the path(s) where your graphic files are
  % \graphicspath{{../pdf/}{../jpeg/}}
  % and their extensions so you won't have to specify these with
  % every instance of \includegraphics
  % \DeclareGraphicsExtensions{.pdf,.jpeg,.png}
\else
  % or other class option (dvipsone, dvipdf, if not using dvips). graphicx
  % will default to the driver specified in the system graphics.cfg if no
  % driver is specified.
  % \usepackage[dvips]{graphicx}
  % declare the path(s) where your graphic files are
  % \graphicspath{{../eps/}}
  % and their extensions so you won't have to specify these with
  % every instance of \includegraphics
  % \DeclareGraphicsExtensions{.eps}
\fi

\usepackage{graphicx}
\usepackage[cmex10]{amsmath}
\usepackage{algorithmic}
\usepackage{array}
\usepackage{mdwmath}
\usepackage{mdwtab}
\usepackage{eqparbox}
%\usepackage[tight,footnotesize]{subfigure}
%\usepackage[caption=false]{caption}
%\usepackage[font=footnotesize]{subfig}
%\usepackage[caption=false,font=footnotesize]{subfig}
\usepackage{fixltx2e}
\usepackage{stfloats}
\usepackage{url}

% *** Do not adjust lengths that control margins, column widths, etc. ***
% *** Do not use packages that alter fonts (such as pslatex).         ***
% There should be no need to do such things with IEEEtran.cls V1.6 and later.
% (Unless specifically asked to do so by the journal or conference you plan
% to submit to, of course. )

% correct bad hyphenation here
\hyphenation{op-tical net-works semi-conduc-tor}

\begin{document}
%
% paper title
% can use linebreaks \\ within to get better formatting as desired
\title{TODO: A Survey of StarCraft AI Techniques}
%
%
% author names and IEEE memberships
% note positions of commas and nonbreaking spaces ( ~ ) LaTeX will not break
% a structure at a ~ so this keeps an author's name from being broken across
% two lines.
% use \thanks{} to gain access to the first footnote area
% a separate \thanks must be used for each paragraph as LaTeX2e's \thanks
% was not built to handle multiple paragraphs
%

\author{FirstName~LastName,~\IEEEmembership{Member,~IEEE,}
        Jim~Raynor,~\IEEEmembership{Fellow,~RR,}
        and~Sarah~Kerrigan,~\IEEEmembership{Life~Fellow,~ZS}% <-this % stops a space
\thanks{FirstName~LastName is with the Department of Names
GA, 30332 USA e-mail: (see http://www.michaelshell.org/contact.html).}% <-this % stops a space
\thanks{J. Raynor and S. Kerrigane are with the Romeo\&Juliet Inc.}% <-this % stops a space
\thanks{Manuscript received April 19, 2499; revised January 11, 2500.}}

% note the % following the last \IEEEmembership and also \thanks - 
% these prevent an unwanted space from occurring between the last author name
% and the end of the author line. i.e., if you had this:
% 
% \author{....lastname \thanks{...} \thanks{...} }
%                     ^------------^------------^----Do not want these spaces!
%
% a space would be appended to the last name and could cause every name on that
% line to be shifted left slightly. This is one of those "LaTeX things". For
% instance, "\textbf{A} \textbf{B}" will typeset as "A B" not "AB". To get
% "AB" then you have to do: "\textbf{A}\textbf{B}"
% \thanks is no different in this regard, so shield the last } of each \thanks
% that ends a line with a % and do not let a space in before the next \thanks.
% Spaces after \IEEEmembership other than the last one are OK (and needed) as
% you are supposed to have spaces between the names. For what it is worth,
% this is a minor point as most people would not even notice if the said evil
% space somehow managed to creep in.

% The paper headers
\markboth{TCIAIG ~Vol.~X, No.~Y, Month~Year}%
{Shell \MakeLowercase{\textit{et al.}}: TODO: Title here}

\maketitle

\begin{abstract}
TODO
\end{abstract}

\begin{IEEEkeywords}
review, RTS, StarCraft, machine learning, planning, TODO ...

\end{IEEEkeywords}

% For peer review papers, you can put extra information on the cover
% page as needed:
% \ifCLASSOPTIONpeerreview
% \begin{center} \bfseries EDICS Category: 3-BBND \end{center}
% \fi
%
% For peerreview papers, this IEEEtran command inserts a page break and
% creates the second title. It will be ignored for other modes.
\IEEEpeerreviewmaketitle

\section{Introduction}
\IEEEPARstart{S}{tarcraft} AI competitions have caused many AI techniques to be
applied to RTS AI. We will list and classify these approaches, explain their 
power and their downsides and conclude on what is left to achieve human-level 
RTS AI. TODO (test \cite{WeberCig10})

\section{Challenges, why is it hard to do a good RTS AI?}

{\color{red} Here we can use Buro's 2003 paper as a starting point. Much has changed since, so we should update, and put his predictions in perspective}

\section{AI Techniques review}
\subsection{Overview}
\subsection{Case study 1: EISBot}
\subsection{Case study 2: NOVA}
\subsection{Case study 3: BroodwarBotQ}

\section{AI Architectures for RTS AI}\label{sec:architecture}

{\color{red}[SANTI: This section is WIP, so, don't read it yet :)]}

Playing an RTS game involves dealing with all the problems described above. A few approaches, like CAT \cite{aha2005cat}, Darmok \cite{ontanon2010darmok} or XXX \cite{some-rl-system} ({\color{red}Which was that system by Stuart Russell that played Warcraft?}) try to deal with the problem in a monolithic manner, by using a single AI technique. This resembles approaches to solve other games, such as Chess or Go, where a single game-tree search approach is enough to play the game at human level. However, none of those systems aims at achieving near human performance. In order to achieve human-level performance, RTS AI designers use a lot of domain knowledge in order to divide the task of playing the game into a collection of sub-problems, which can be dealt-with using individual AI techniques (as discussed in the previous section). Thus, an integration architecture is required to put together all of those techniques into a single coherent system, which is the focus of this section.

...

\begin{figure}[ta]
%    \centering
    \includegraphics[width=\columnwidth]{figures/figure-bot-architectures.pdf}
    \caption{General architecture of 5 Starcraft AI bots. {\color{red} (This figure is unreadable as of now, but we'll figure out a way to better present this info)}}
    \label{fig:bot-architecture}
\end{figure}

{\color{red} Present some representative examples (Figure \ref{fig:bot-architecture}), which have slightly different architectures. I could mention THANATOS as something different using learning, and Darmok as a monolithic architecture that aims at learning how to play the game, without hoping to play at human level.}

{\color{red} Then, draw some general conclusions: 
\begin{itemize}
\item two types of aggregation, hierarchical, and decentralized (bidding?). Hierarchical is multiscale, with actions at different levels of abstraction.
\item high level strategy is hard-coded (unlike in games like Chess or Go): different expressivity of the high-level strategy of the different systems, compare them. It's like a very high level programming language for RTS bots.
\end{itemize}}


\section{Discussion, what is left to do / open challenges}

\section{Conclusion}





\section*{Acknowledgment}

% Can use something like this to put references on a page
% by themselves when using endfloat and the captionsoff option.
\ifCLASSOPTIONcaptionsoff
  \newpage
\fi

%\begin{thebibliography}{1}
%\end{thebibliography}
\bibliographystyle{IEEEtran}                                                    
\bibliography{survey}

%\begin{IEEEbiographynophoto}{FirstName LastName}
%Biography text here.
%\end{IEEEbiographynophoto}

\begin{IEEEbiography}[{\includegraphics[width=2cm, keepaspectratio]{jim.jpg}}]{Jim Raynor}
Jim Raynor was a Confederate marshal on Mar Sara at the time of the first zerg incursions on that world. He is now with Raynor's Raiders Inc.
\end{IEEEbiography}

\end{document}


